\documentclass[11pt]{report}   % list options between brackets
\usepackage{parskip}
\usepackage[T1]{fontenc}
\usepackage{cmbright}
\usepackage{hyperref}
\hypersetup{
  colorlinks=true,
  linkcolor=blue,
  urlcolor=cyan,
}

% type user-defined commands here

\begin{document}

\title{\TeX\ and \LaTeX}   % type title between braces
\author{Amy Roberts}         % type author(s) between braces
\date{October 27, 1995}    % type date between braces
% \maketitle

%\begin{abstract}
%  A brief introduction to \TeX\ and \LaTeX
%\end{abstract}

%\section*{The problem}     % section 1.1
\section*{Amy Roberts \\ CSSI Proposal Summary}
Taking science data in nuclear physics and other medium-count-rate disciplines often requires creating highly-customized data acquisition software - and this represents an enormous loss of time for the scientific community.  Although the data acquisition constraints and the final data on disk vary substantially, many scientists have shared needs.  At minimum, we all need easy access to our data - which often must be stored in a custom format.  Over and over, scientists write code to read their data, sinking isolated time into solving a common problem.

% of Getting critical experimental data often involves highly-customized equipment - and for scientists developing new detectors this
%and because writing a custom data acquisition system requires significant investment from hig
%hly-trained experts, scientists at smaller institutions without dedicated lab staff are often
%cut off from detector development.

%from scientists who are studying fusion and have to build custom detectors and electronics to
%T handle the extraordinarily high rates of data to scientists who are developing detectors sens
%itive enough to "hear" neutrinos and have to build custom electronics to give extremely low no
%ise performance.  A


I propose to (1) define a core set of data-acquisition standards and (2) develop standards-based tools that are immediately useful to scientists taking and analyzing data.  One example is a standard data-description language, along with a tool that could use a scientist's format description to provide basic data-reading and data-writing utilities.  A tool that provides convenient access to data while requiring scientists only to describe their custom format in a standard way would be immediately useful at university-based experimental facilities and collaborations.  

Standards allowing the medium-count-rate community to share work are increasingly needed as we move towards detector systems with multiple, specialized sensors.  In addition, standards would make it significantly more likely that small, independent labs - which are less often able to support staff dedicated to software support - would be able to use community tools for their data acquisition needs.  

\end{document}
